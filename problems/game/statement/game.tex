\begin{problem}{Метро}{stdin}{stdout}{1 секунда}{512 мегабайт}

<<Многие>> \t{знают}, \w{что} метро это самый удобный и быстрый общественный транспорт (конечно там, где оно вообще есть). Но к нашему главному герою изобретателю Акакию пришла замечательная мысль, а именно проект под названием <<Метро будущего>>. Он уже давно заметил, что время проведенное в метро прямо пропорционально длине ветки, так как с ростом длины веток растет и время, затрачиваемое поездом на то, чтобы проехать по этой ветке. Поэтому он в своем проекте решил свести длину путей к минимуму. И в результате долгих размышлений он пришел к выводу, что метро должно состоять из одной станции, и тогда суммарная длина веток будет равна нулю. Акакий хотел бы уменьшить ее еще сильней, но у него пока что не получилось. Но, так как метро должно быть доступно из любой точки города, он решил сделать еще несколько спусков под землю и соединить их между собой и со станцией подземными переходами (у нас даже есть информация, что в одном из городов России этот проект уже воплощают в жизнь). 

Акакий создал идеальный план метрополитена так, чтобы из каждого спуска можно было добраться до каждого, но строители были мигрантами из какой-то далекой страны, где слово <<метро>> обозначает подземелье, созданное для того, чтобы запускать туда людей и пытать их. Поэтому они постарались сделать его как можно лучше, и в эксплуатацию пошло метро без части переходов. Акакий был в ярости, когда узнал про это, но ничего не поделаешь нужно было учитывать особенности перевода... 

Теперь он хочет знать лишь то, можно ли добраться от спуска около его дома до спуска рядом с его работой по подземным переходам. Но схема настолько сложна, что вручную ее не разобрать, поэтому он решил обратиться к вам за помощью. Ах да, Акакий уже гражданин преклонных лет, из-за чего он уже не помнит, где живет и работает, поэтому вам передадут $t$ запросов, для каждого из которых вам нужно будет дать ответ. 

\InputFile
В первой строке входного файла заданы два числа $n$ и $m$ ($1 \le n \le 1000, 1 \le m \le \frac{n(n-1)}{2}$)~--- количество спусков под землю и количество переходов.

Далее, в $m$ строках идут числа $f_i$ и $t_i$ ($1 \le f_i, t_i \le n$), номера спусков, который этот переход соединяет. Сама станция совмещена со спуском 1, который остался после ее строительства.

Далее идет число $t$~--- количество запросов ($t \le 10^5$). Далее в каждый запрос задается двумя числами --- начальным и конечным спуском.

\OutputFile
Для $i$-го запроса в $i$-ой строке выведите "YES" (без кавычек), если Акакий сможет добраться от дома до работы, иначе "NO".
\Examples

\begin{example}%
\exmp{
6 4
1 2
1 3
2 3
5 4
5
3 2
6 1
4 5
1 5
1 2
}{
YES
NO
YES
NO
YES
}%
\end{example}

\Note
Тесты к этой задаче состоят из пяти групп:
\begin{itemize}
\item Тест $1$. Тест из условия. Оценивается в 0 баллов.
\item Тесты $2-16$. В тестах этой группы к каждому спуску проведено не более двух переходов. Переходы не могут образовывать циклы. $t \leq 10$. Оцениваются в 30 баллов.
\item Тесты $17-31$. В тестах этой группы к каждому спуску проведено не более двух переходов. $t \leq 10$. Оцениваются в 30 баллов.
\item Тесты $31-41$. В тестах этой группы $n \leq 100, t \leq 10$. Оцениваются в 20 баллов.
\item Тесты $41-51$. В тестах этой группы дополнительные ограничения отсутствуют. Оцениваются в 20 баллов.
\end{itemize}

\end{problem}

