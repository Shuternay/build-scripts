\begin{problem}{Метро}{}{}{1 секунда}{512 мегабайт}

Теперь он хочет знать лишь то, можно ли добраться от спуска около его дома до спуска рядом с его работой по подземным переходам. Но схема настолько сложна, что вручную ее не разобрать, поэтому он решил обратиться к вам за помощью. Ах да, Акакий уже гражданин преклонных лет, из-за чего он уже не помнит, где живет и работает, поэтому вам передадут $t$ запросов, для каждого из которых вам нужно будет дать ответ. 

\InputFile
В первой строке входного файла заданы два числа $n$ и $m$ ($1 \le n \le 1000, 1 \le m \le \frac{n(n-1)}{2}$)~--- количество спусков под землю и количество переходов.

Далее, в $m$ строках идут числа $f_i$ и $t_i$ ($1 \le f_i, t_i \le n$), номера спусков, который этот переход соединяет. Сама станция совмещена со спуском 1, который остался после ее строительства.

Далее идет число $t$~--- количество запросов ($t \le 10^5$). Далее в каждый запрос задается двумя числами --- начальным и конечным спуском.

\OutputFile
Для $i$-го запроса в $i$-ой строке выведите "YES" (без кавычек), если Акакий сможет добраться от дома до работы, иначе "NO".

\Examples

\begin{example}%
\exmp{
1 2
}{
YES
}%
\end{example}

\Explanations
Слово из первого примера было получено так: \w{abc} -> \w{ab~def~c} -> \w{abde~ghk~fc}.

\Note
Тесты к этой задаче состоят из нескольких групп:

\begin{itemize}
\setlength{\itemsep}{-1mm}
\item Группа 0 (0 баллов). Тест $1-2$. Тесты из условия.
\item Группа 1 (40 баллов). Тесты $3-12$. В тестах этой группы длина строки не больше $10^3$.
\item Группа 2 (40 баллов). Тесты $13-22$. В тестах этой группы длина строки не больше $10^4$.
\item Группа 3 (20 баллов). Тесты $23-27$. В тестах этой группы дополнительные ограничения отсутствуют.
\end{itemize}


\end{problem}


